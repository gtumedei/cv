\documentclass[a4paper,10pt]{article}

% This entry point is for an attachment to PhD calls

% Packages
% -------------------- %

% Apply the document encoding
\usepackage[utf8]{inputenc}
% Set the document language/s
\usepackage[english]{babel}
% Set margins
\usepackage[hmargin=2cm,vmargin=3cm,headheight=16pt]{geometry}
% Apply the Merryweather font
\usepackage{merriweather}
\usepackage[T1]{fontenc}
% Customize links
\usepackage[english]{hyperref}
% Include graphics
\usepackage{graphicx}
% Create better tables
\usepackage{tabularx}
% Use floating positioning (like H inside figures)
\usepackage{float}
% Customize lists
\usepackage{enumitem}
% Customize the line height
\usepackage{setspace}
% Customize the space between paragraphs
\usepackage{parskip}
% Provide additional color syntax
\usepackage{xcolor}
% Customize headers and footers
\usepackage{fancyhdr}
% Create cells inside tables
\usepackage{makecell}
% Customize section style
\usepackage{titlesec}

% Package config
% -------------------- %

% Set line height
\setstretch{1.2}
% Remove indent for the first paragraph line
\setlength{\parindent}{0pt}
% Set space between paragraphs
\setlength{\parskip}{6pt}

% Remove table cell padding
\renewcommand{\tabcolsep}{1pt}

% Customize link colors
\hypersetup{
  colorlinks=true,
  linkcolor=black,
  filecolor=black,
  urlcolor=black,
  citecolor=black
}
\urlstyle{same}

% Use dashes for itemize
\renewcommand\labelitemi{-}

% Remove space between itemize elements and decrease the left margin
\setlist[itemize]{nosep,leftmargin=*}

% Set graphics folder path
\graphicspath{{./img/}}

% Heading color
\definecolor{sky}{HTML}{2079C7}

% Setup page styles

% Page number only
\fancypagestyle{style1}{
  \fancyhf{}
  \renewcommand{\headrulewidth}{0pt}
  \fancyfoot[R]{\thepage}
}

% Name and page number
\fancypagestyle{style2}{
  \fancyhf{}
  \renewcommand{\headrulewidth}{0pt}
  \fancyhead[R]{\name}
  \fancyfoot[R]{\thepage}
}

% Hide section numbers without breaking everything
\renewcommand\thesection{}
\makeatletter
\def\@seccntformat#1{\csname #1ignore\expandafter\endcsname\csname the#1\endcsname\quad}
\let\sectionignore\@gobbletwo
\let\latex@numberline\numberline
\def\numberline#1{\if\relax#1\relax\else\latex@numberline{#1}\fi}
\makeatother

% Setup section style
\titleformat{\section}[hang]
  {\sffamily\bfseries\color{sky}}{\thesection}{0em}{\MakeUppercase}

\newcommand{\name}{Gianni Tumedei}

\renewcommand{\title}[1]{{\huge \textbf{#1}}}

\renewcommand{\section}[1]{{\sffamily
  \vspace{10pt}
  \textcolor{sky}{\textbf{\expandafter\MakeUppercase\expandafter{#1}}}
  \vspace{20pt}
}}

\newcommand{\heading}[1]{{\large \textbf{#1} \vspace{4pt}}}

\newcommand{\subheading}[1]{\textit{#1}\newline}

\newcommand{\twocols}[2]{
  \begin{tabularx}{\textwidth}{ p{0.3\textwidth} X p{0.65\textwidth} }
    \makecell[tl]{#1} && #2 \\
  \end{tabularx}
  \vspace{20pt}
}

\input{env}

\begin{document}

\pagestyle{style1}

\title{\name}

\section{University-Level Teaching}

\twocols{
  2022 - Today
}{
  \heading{University of Bologna}\newline
  \subheading{Teaching Tutor}

  \textbf{Subject}: Web Services and Applications (Applicazioni e Servizi Web)\newline
  \textbf{Course}: second cycle degree in Computer Science and Engineering (Ingegneria e Scienze Informatiche)\newline
  \textbf{Tasks}:
  \begin{itemize}
    \item Hold lectures
    \item Support in laboratory lectures
  \end{itemize}
}

\twocols{
  2022 - Today
}{
  \heading{University of Bologna}\newline
  \subheading{Teaching Tutor}

  \textbf{Subject}: Web Systems Engineering (Ingegneria dei Sistemi Web)\newline
  \textbf{Course}: first cycle degree in Computer Systems Technologies (Tecnologie dei Sistemi Informatici)\newline
  \textbf{Tasks}:
  \begin{itemize}
    \item Hold lectures
    \item Support in laboratory lectures
    \item Correction of exam projects
  \end{itemize}
}

\twocols{
  2022
}{
  \heading{University of Bologna}\newline
  \subheading{Teaching Tutor}

  \textbf{Subject}: Foundations of Web Systems (Fondamenti di Sistemi Web)\newline
  \textbf{Course}: first cycle degree in Computer Systems Technologies (Tecnologie dei Sistemi Informatici)\newline
  \textbf{Tasks}:
  \begin{itemize}
    \item Support in laboratory lectures
    \item Correction of exam projects
  \end{itemize}
}

\section{Scientific Research Activities}

\twocols{
  2023 - Today
}{
  \heading{University of Bologna}\newline
  \subheading{Research Fellow}

  \textbf{Project}: H2020\_SMARTLAGOON\newline
  \textbf{Subject}: Designing Digital Twin interfaces and Data Visualization for increasing the stakeholders' awareness in the SMARTLAGOON context\newline
  \textbf{Description}: The project related to this position is within the context of the H2020 FET SMARTLAGOON, with the main objective of developing the Mar Menor digital twin that can predict future problems and provide stakeholders with a tool for sustainable decision-making. The motivation behind this approach is that coastal lagoons are ecosystems of great environmental and socio-economic value; however, they are also particularly vulnerable to climatic and anthropogenic pressures, such as intensive agriculture and extensive urbanisation due to tourism development. The vulnerability and complexity of these ecosystems require a systemic understanding of the socio-environmental interrelationships affecting coastal lagoons and their ecosystem. Real-time monitoring, analysis and management of these critical resources can provide a mechanism for finding new solutions to assess the trade-offs between socio-economic and environmental aspects. Such solutions must be based on the analysis of data from different sources, both social (e.g. reports and questionnaires, social-media, legislation) and physical (e.g. meteorological, historical). The analysis can be carried out through the creation of a digital twin. Nevertheless, given the complexity of the data and the large volume, it becomes essential to design the digital twin's interfaces in such a way that the data is comprehensible to the various stakeholders.
}

\twocols{
  May - September 2021
}{
  \heading{University of Bologna - Navile Project}\newline
  \subheading{Contract for research activity}

  Development, for the new Navile building complex of the University of Bologna, of an intuitive and usable wayfinding platform that provides public kiosk displays with a responsive UI, SVG maps and indications on how to reach campus spaces.\newline

  \textbf{Supervisor}: Prof. Catia Prandi\newline
  \textbf{Research team}: two researchers of the Computer Science and Engineering department, two research students\newline
  \textbf{Technologies}: Node.js, TypeScript, Vue.js, Express, MySQL
}

\twocols{
  December 2020 - February 2021
}{
  \heading{University of Lisbon - Maré Project}\newline
  \subheading{Scholarship for research activities}

  Development of a platform to help students return to normal after Covid-19, by offering a ML predicted value of the number of people inside campus spaces, a feedback system to improve estimates, and an unenforced booking system, all accessible via a mobile application.\newline

  \textbf{Supervisors}: Prof. Catia Prandi, Prof. Augusto Esteves\newline
  \textbf{Research team}: international team with members from the University of Bologna, University of Minho, University of Porto, ITI / LARSyS, University of Lisbon\newline
  \textbf{Technologies}: Dart, Flutter, Google Maps SDK, Firebase\newline
  \textbf{Related publications}:
  \begin{itemize}
    \item Ma, S., Nisi, V., Esteves, A., Prandi, C., Nicolau, H., \textbf{Tumedei, G.}, Nogueira, J., Boschi, F. and Nunes, N., 2022. Crowdsensing-enabled service design for floating students during the COVID-19 pandemic. In \textit{Congress of the International Association of Societies of Design Research} (pp. 943-959). Springer, Singapore.
    \item \textbf{Tumedei, G.}, Boschi, F., Prandi, C., Gomes, L., Calheno, R., Abreu, R., Ma, S., Nisi, V., Nunes, N. and Esteves, A., 2021, September. Promoting a safe return to university campuses during the COVID-19 pandemic: Crowdsensing room occupancy. In \textit{Proceedings of the Conference on Information Technology for Social Good} (pp. 145-150).
  \end{itemize}
}

\section{Awards and Acknowledgements}

\twocols{
  2021
}{
  \heading{OSM11 Hackfest - Winner}\newline
  \subheading{Team Asterisk Unibo: Onur Ozenir, Gianni Tumedei, Gyordan Caminatti, Luca Salvigni, Mattia Rossi}

  Exploiting OpenStack, Juju Charms and Asterisk for the implementation and deployment of a Virtual Network Function for VoIP calls.\newline

  \link{https://osm.etsi.org/wikipub/index.php/OSM11_Hackfest}{https://osm.etsi.org/wikipub/index.php/OSM11\_Hackfest}
}

\section{Scientific Conferences}

\twocols{
  17-21 July 2024
}{
  \heading{The 22nd European Conference on Computer-Supported Cooperative Work (ECSCW2024)}\newline
  \subheading{Web Chair}
}

\twocols{
  10-13 July 2022
}{
  \heading{IEEE International Conference on Distributed Computing Systems (ICDCS 2022)}\newline
  \subheading{Volunteer}
}

\twocols{
  09-11 September 2021
}{
  \heading{ACM International Conference on Information Technology for Social Good (GoodIT 2021)}\newline
  \subheading{Speaker}

  \textbf{Track title}: Promoting a Safe Return to University Campuses during the COVID-19 Pandemic: Crowdsensing Room Occupancy
}


\end{document}
