\section{Work Experience}

\twocols{
  November 2023 - Today
}{
  \heading{University of Bologna}\newline
  \subheading{PhD Student}

  \textbf{Research topic}: Digital Twin and HCI to Augment Smart Environment

  \textbf{Description}: This PhD project's goal is to apply HCI techniques to visualize and manipulate data within Digital Twins (DTs), investigating on the effectiveness of classic user interfaces, eXtended Reality (XR), voice commands and more. By focusing on some case studies, mainly about smart environment digital twins, the idea is to find out how the human-DT interface can be designed to positively impact aspects like infrastructure management, awareness and decision-making processes. The end results are going to be generalized into a set of universal guidelines on how to communicate and manipulate DT data across diverse fields.
}

% \twocols{
%   November - December 2023
% }{
%   \heading{Radio Centrale}\newline
%   \subheading{Web developer}

%   \textbf{Description}: creation of the new Radio Centrale website.

%   \textbf{Website}: \link{https://radiocentraleweb.it}{https://radiocentraleweb.it}

%   \textbf{Technologies}: Node.js, TypeScript, Astro, Solid.js, Payload CMS, MongoDB.
% }

\twocols{
  a.y. 2023/2024
}{
  \heading{University of Bologna}\newline
  \subheading{Teaching Tutor}

  \textbf{Subject}: Mobile Systems Programming (Programmazione di Sistemi Mobile)

  \textbf{Course}: first cycle degree in Computer Science and Engineering (Ingegneria e Scienze Informatiche)

  \textbf{Tasks}:
  \begin{itemize}
    \item Hold lectures and laboratory exercises
  \end{itemize}
}

\twocols{
  February - October 2023
}{
  \heading{University of Bologna}\newline
  \subheading{Research Fellow}

  \textbf{Project}: H2020\_SMARTLAGOON

  \textbf{Subject}: Designing Digital Twin interfaces and Data Visualization for increasing the stakeholders' awareness in the SMARTLAGOON context
}

\twocols{
  a.y. 2022/2023
}{
  \heading{University of Bologna}\newline
  \subheading{Teaching Tutor}

  \textbf{Subject}: Web Services and Applications (Applicazioni e Servizi Web)

  \textbf{Course}: second cycle degree in Computer Science and Engineering (Ingegneria e Scienze Informatiche)

  \textbf{Tasks}:
  \begin{itemize}
    \item Hold lectures
    \item Support in-laboratory lectures
  \end{itemize}
}

\twocols{
  a.y. 2022/2023
}{
  \heading{University of Bologna}\newline
  \subheading{Teaching Tutor}

  \textbf{Subject}: Web Systems Engineering (Ingegneria dei Sistemi Web)

  \textbf{Course}: first cycle degree in Computer Systems Technologies (Tecnologie dei Sistemi Informatici)

  \textbf{Tasks}:
  \begin{itemize}
    \item Hold lectures
    \item Support in-laboratory lectures
    \item Correction of exam projects
  \end{itemize}
}

\twocols{
  a.y. 2021/2022
}{
  \heading{University of Bologna}\newline
  \subheading{Teaching Tutor}

  \textbf{Subject}: Foundations of Web Systems (Fondamenti di Sistemi Web)

  \textbf{Course}: first cycle degree in Computer Systems Technologies (Tecnologie dei Sistemi Informatici)

  \textbf{Tasks}:
  \begin{itemize}
    \item Support in-laboratory lectures
    \item Correction of exam projects
  \end{itemize}
}

\twocols{
  October - December 2021
}{
  \heading{Scientific High School “Augusto Righi”, Cesena (FC)}\newline
  \subheading{Computer Science and Technology Teacher}

  \textbf{Tasks}:
  \begin{itemize}
    \item Substitute teacher in the subject of Computer Science
    \item Support for foreign students in learning the italian language and studying Computer Science
  \end{itemize}
}

\twocols{
  May - September 2021
}{
  \heading{University of Bologna - Navile Project}\newline
  \subheading{Contract for research activity}

  Development, for the new Navile building complex of the University of Bologna, of an intuitive and usable wayfinding platform that provides public kiosk displays with a responsive UI, SVG maps and indications on how to reach campus spaces.

  \textbf{Supervisor}: Prof. Catia Prandi

  \textbf{Research team}: two researchers of the Computer Science and Engineering department, two research students

  \textbf{Technologies}: Node.js, TypeScript, Vue.js, Express, MySQL
}

\twocols{
  2021
}{
  \heading{Morethantech}\newline
  \subheading{Web developer}

  Creation and maintenance of the Morethantech website.

  \textbf{Website}: \link{https://morethantech.it}{https://morethantech.it}

  \textbf{Technologies}: Node.js, TypeScript, Vue.js, tRPC, MySQL, Docker.
}

\twocols{
  December 2020 - February 2021
}{
  \heading{University of Lisbon - Maré Project}\newline
  \subheading{Scholarship for research activities}

  Development of a platform to help students return to normal after Covid-19, by offering a ML predicted value of the number of people inside campus spaces, a feedback system to improve estimates, and an unenforced booking system, all accessible via a mobile application.

  \textbf{Supervisors}: Prof. Catia Prandi, Prof. Augusto Esteves

  \textbf{Research team}: international team with members from the University of Bologna, University of Minho, University of Porto, ITI / LARSyS, University of Lisbon

  \textbf{Technologies}: Dart, Flutter, Google Maps SDK, Firebase

  \textbf{Related publications}:
  \begin{itemize}
    \item Ma, S., Nisi, V., Esteves, A., Prandi, C., Nicolau, H., \textbf{Tumedei, G.}, Nogueira, J., Boschi, F. and Nunes, N., 2022. Crowdsensing-enabled service design for floating students during the COVID-19 pandemic. In \textit{Congress of the International Association of Societies of Design Research} (pp. 943-959). Springer, Singapore.
    \item \textbf{Tumedei, G.}, Boschi, F., Prandi, C., Gomes, L., Calheno, R., Abreu, R., Ma, S., Nisi, V., Nunes, N. and Esteves, A., 2021, September. Promoting a safe return to university campuses during the COVID-19 pandemic: Crowdsensing room occupancy. In \textit{Proceedings of the Conference on Information Technology for Social Good} (pp. 145-150).
  \end{itemize}
}

\twocols{
  October - November 2019
}{
  \heading{Tajana, Barlocco, Galluccio \& Partners}\newline
  \subheading{Web developer}

  Creation of a new website for the company.

  \textbf{Website}: \link{https://tbgstudio.it}{https://tbgstudio.it}

  \textbf{Technologies}: HTML, CSS, JavaScript, Bootstrap, PHP.
}

\twocols{
  April 2018
}{
  \heading{Infia S.r.l.}\newline
  \subheading{Occasional performance}

  VBA macro programming in Microsoft Excel.
}

\twocols{
  October 2016 - July 2017
}{
  \heading{Infia S.r.l.}\newline
  \subheading{Employment - IT support}

  \textbf{Activities and responsibilities}:
  \begin{itemize}
    \item Supporting the IT manager
    \item Troubleshooting, maintenance and system integration and configuration tasks
    \item Collaboration in the migration project towards ERP and MES systems, with particular regard to data migration
    \item Collaboration in the project for the installation of an automatic palletizer in one of the company's production departments
    \item IT assistance to employees
    \item Employees workstation setup
    \item VBA macro programming in Microsoft Excel
    \item Queries on SQL Server Management Studio
    \item Product label design using ZPL language
  \end{itemize}
}

\twocols{
  February - July 2016
}{
  \heading{Infia S.r.l.}\newline
  \subheading{Internship - IT support}

  \textbf{Activities and responsibilities}:
  \begin{itemize}
    \item Supporting the IT manager in various troubleshooting, maintenance and system integration and configuration tasks
    \item IT assistance to employees
    \item Employees workstation setup
    \item VBA macro programming in Microsoft Excel
    \item Queries on SQL Server Management Studio
  \end{itemize}
}

\twocols{
  June - July 2014
}{
  \heading{Infia S.r.l.}\newline
  \subheading{Internship in collaboration with ITT B. Pascal}

  \textbf{Activities and responsibilities}:
  \begin{itemize}
    \item Supporting the IT manager
    \item IT assistance to employees
    \item VBA macro programming in Microsoft Excel
    \item Queries on SQL Server Management Studio
  \end{itemize}
}

\twocols{
  July 2013, July - August 2012
}{
  \heading{eSTATE ATTIVI project - ITAS G. Garibaldi Cesena}\newline
  \subheading{Volunteering il collaboration with the Municipality of Cesena}

  Sale of the company's fruit products.
}
