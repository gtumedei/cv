\section{Education and Qualifications}

\twocols{
  November 2023 - Today
}{
  \heading{University of Bologna}\newline
  \subheading{PhD Student}

  \textbf{Research topic}: Digital Twin and HCI to Augment Smart Environment

  \textbf{Description}: This PhD project's goal is to apply HCI techniques to visualize and manipulate data within Digital Twins (DTs), investigating on the effectiveness of classic user interfaces, eXtended Reality (XR), voice commands and more. By focusing on some case studies, mainly about smart environment digital twins, the idea is to find out how the human-DT interface can be designed to positively impact aspects like infrastructure management, awareness and decision-making processes. The end results are going to be generalized into a set of universal guidelines on how to communicate and manipulate DT data across diverse fields.
}

\twocols{
  2019 - 2022
}{
  \heading{Second cycle degree in Computer Science and Engineering}\newline
  \subheading{University of Bologna, Campus of Cesena}

  \textbf{Grade}: 110/110 and honors

  \textbf{Class n.}: LM-32 Computer systems engineering

  \textbf{Elaborate details}:
  \begin{itemize}[itemsep=4pt]
    \item[] \textbf{Title}: Towards a Smart Campus Digital Twin: Promoting Awareness and Sustainability Through Wayfinding and Real-Time Environmental Data
    \item[] \textbf{Subjects}: Web Services and Applications
    \item[] \textbf{Supervisor}: Prof. Catia Prandi
    \item[] \textbf{Description}: Documentation of the AlmaMap Navile project, a system that powers public touch kiosks across the Navile district of the University of Bologna, offering two main features: i) maps and wayfinding to help people in orienting themselves; and ii) real-time environmental data and comfort level about campus spaces, obtained by interfacing with the BIM system installed at the complex. The end result is a first step towards building a smart campus Digital Twin, that can support all the people who are part of the campus life in their daily activities, improving their experience, giving them the ability to make informed decisions, and promoting awareness and sustainability.
    \item[] \textbf{URL}: \link{https://amslaurea.unibo.it/27643}{https://amslaurea.unibo.it/27643}
  \end{itemize}
}

\twocols{
  2015 - 2019
}{
  \heading{First cycle degree in Computer Science and Engineering}\newline
  \subheading{University of Bologna, Campus of Cesena}

  \textbf{Grade}: 100/110

  \textbf{Class n.}: L-8 Information technology engineering

  \textbf{Elaborate details}:
  \begin{itemize}[itemsep=4pt]
    \item[] \textbf{Title}: Il protocollo OPC UA
    \item[] \textbf{Supervisor}: Prof. Franco Callegati
    \item[] \textbf{Subject}: Telecommunication Networks
    \item[] \textbf{Description}: Detailed analysis of the OPC Unified Architecture protocol, an open source and cross platform solution for data exchange in the industrial and IoT fields.
    \item[] \textbf{URL}: \link{https://amslaurea.unibo.it/19780}{https://amslaurea.unibo.it/19780}
  \end{itemize}
}

\twocols{
  2010 - 2015
}{
  \heading{High school graduation in Computer Science and Telecommunications}\newline
  \subheading{Istituto Tecnico Tecnologico Statale Blaise Pascal - P.le Macrelli, 100 - 47521 Cesena (FC)}

  \textbf{Grade}: 100/100

  \textbf{Address subjects}:
  \begin{itemize}
    \item Computer technology
    \item Systems and Networks
    \item Project Management and Business Organization
    \item Electronics and Telecommunications
    \item Systems Technology \& Design
  \end{itemize}
}

\twocols{
  2015
}{
  \heading{First Certificate in English (FCE)}\newline
  \subheading{Istituto Tecnico Tecnologico Statale Blaise Pascal - P.le Macrelli, 100 - 47521 Cesena (FC)}

  \textbf{Grade}: A

  \textbf{Level}: C1

  \textbf{Certification body}: Cambridge English Language Assessment
}

\twocols{
  2015
}{
  \heading{Garden in-depth and excellence course}\newline
  \subheading{Istituto Tecnico Tecnologico Statale Blaise Pascal - P.le Macrelli, 100 - 47521 Cesena (FC)}

  Creation of a greenhouse equipped with an online data acquisition and consultation system.
}

\twocols{
  2014
}{
  \heading{Basic training course for workers}\newline
  \subheading{Istituto Tecnico Tecnologico Statale Blaise Pascal - P.le Macrelli, 100 - 47521 Cesena (FC)}

  Training on:
  \begin{itemize}
    \item Concepts of risk, damage, prevention, protection
    \item Rights and duties of the various company subjects
    \item Mechanical, electrical, chemical, carcinogenic, biological risks
    \item Workplaces, VDTs, work stress
    \item Signs, emergencies
    \item Safety and first aid procedures
    \item Accidents and injuries
  \end{itemize}
}

\twocols{
  2013
}{
  \heading{Preliminary English Test (PET)}\newline
  \subheading{Istituto Tecnico Tecnologico Statale Blaise Pascal - P.le Macrelli, 100 - 47521 Cesena (FC)}

  \textbf{Level}: B2

  \textbf{Certification body}: Cambridge English Language Assessment
}

\twocols{
  2010
}{
  \heading{Key English Test (KET)}\newline
  \subheading{University of Bologna, Campus of Cesena}

  \textbf{Level}: A2

  \textbf{Certification body}: Cambridge English Language Assessment
}
