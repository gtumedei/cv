\section{Education and Qualifications}

\twocols{
  2019 - 2022
}{
  \heading{Second cycle degree in Computer Science and Engineering}\newline
  \subheading{University of Bologna, Campus of Cesena}

  \textbf{Grade}: 110/110 and honors\newline
  \textbf{Class n.}: LM-32 Computer systems engineering\newline
  \textbf{Elaborate details}:
  \begin{itemize}
    \item[] \textbf{Title}: Towards a Smart Campus Digital Twin: Promoting Awareness and Sustainability Through Wayfinding and Real-Time Environmental Data
    \item[] \textbf{Subjects}: Web Services and Applications
    \item[] \textbf{Supervisor}: Prof. Catia Prandi
    \item[] \textbf{Description}: The newly inaugurated Navile District of the University of Bologna is a complex created along the Navile canal, that now houses various teaching and research activities for the disciplines of Chemistry, Industrial Chemistry, Pharmacy, Biotechnology and Astronomy. A Building Information Modeling system (BIM) gives staff of the Navile campus several ways to monitor buildings in the complex throughout their life cycle, one of which is the ability to access real-time environmental data such as room temperature, humidity, air composition, and more, thereby simplifying operations like finding faults and optimizing environmental resource usage. But smart features at Navile are not only available to the staff: AlmaMap Navile is a web application, whose development is documented in this thesis, that powers the public touch kiosks available throughout the campus, offering maps of the district and indications on how to reach buildings and spaces. Even if these two systems, BIM and AlmaMap, don't seem to have many similarities, they share the common intent of promoting awareness for informed decision making in the campus, and they do it while relying on web standards for communication. This opens up interesting possibilities, and is the idea behind AlmaMap Navile 2.0, an app that interfaces with the BIM system and combines real-time sensor data with a comfort calculation algorithm, giving users the ability not just to ask for directions to a space, but also to see its comfort level in advance and, should they want to, check environmental measurements coming from each sensor in a granular manner. The end result is a first step towards building a smart campus Digital Twin, that can support all the people who are part of the campus life in their daily activities, improving their efficiency and satisfaction, giving them the ability to make informed decisions, and promoting awareness and sustainability.
    \item[] \textbf{URL}: \link{https://amslaurea.unibo.it/27643}{https://amslaurea.unibo.it/27643}
  \end{itemize}
}

\twocols{
  2015 - 2019
}{
  \heading{First cycle degree in Computer Science and Engineering}\newline
  \subheading{University of Bologna, Campus of Cesena}

  \textbf{Grade}: 100/110\newline
  \textbf{Class n.}: L-8 Information technology engineering\newline
  \textbf{Elaborate details}:
  \begin{itemize}
    \item[] \textbf{Title}: Il protocollo OPC UA
    \item[] \textbf{Supervisor}: Prof. Franco Callegati
    \item[] \textbf{Subject}: Telecommunication Networks
    \item[] \textbf{Description}: Detailed analysis of the OPC Unified Architecture protocol, an open source and cross platform solution for data exchange in the industrial and IoT fields.
    \item[] \textbf{URL}: \link{https://amslaurea.unibo.it/19780}{https://amslaurea.unibo.it/19780}
  \end{itemize}
}

\twocols{
  2010 - 2015
}{
  \heading{High school graduation in Computer Science and Telecommunications}\newline
  \subheading{Istituto Tecnico Tecnologico Statale Blaise Pascal - P.le Macrelli, 100 - 47521 Cesena (FC)}

  \textbf{Grade}: 100/100\newline
  \textbf{Address subjects}:
  \begin{itemize}
    \item Computer technology
    \item Systems and Networks
    \item Project Management and Business Organization
    \item Electronics and Telecommunications
    \item Systems Technology \& Design
  \end{itemize}
}

\twocols{
  2015
}{
  \heading{First Certificate in English (FCE)}\newline
  \subheading{Istituto Tecnico Tecnologico Statale Blaise Pascal - P.le Macrelli, 100 - 47521 Cesena (FC)}

  \textbf{Grade}: A\newline
  \textbf{Level}: C1\newline
  \textbf{Certification body}: Cambridge English Language Assessment
}

\twocols{
  2015
}{
  \heading{Garden in-depth and excellence course}\newline
  \subheading{Istituto Tecnico Tecnologico Statale Blaise Pascal - P.le Macrelli, 100 - 47521 Cesena (FC)}

  Creation of a greenhouse equipped with an online data acquisition and consultation system.
}

\twocols{
  2014
}{
  \heading{Basic training course for workers}\newline
  \subheading{Istituto Tecnico Tecnologico Statale Blaise Pascal - P.le Macrelli, 100 - 47521 Cesena (FC)}

  Training on:
  \begin{itemize}
    \item Concepts of risk, damage, prevention, protection
    \item Rights and duties of the various company subjects
    \item Mechanical, electrical, chemical, carcinogenic, biological risks
    \item Workplaces, VDTs, work stress
    \item Signs, emergencies
    \item Safety and first aid procedures
    \item Accidents and injuries
  \end{itemize}
}

\twocols{
  2013
}{
  \heading{Preliminary English Test (PET)}\newline
  \subheading{Istituto Tecnico Tecnologico Statale Blaise Pascal - P.le Macrelli, 100 - 47521 Cesena (FC)}

  \textbf{Level}: B2\newline
  \textbf{Certification body}: Cambridge English Language Assessment
}

\twocols{
  2010
}{
  \heading{Key English Test (KET)}\newline
  \subheading{University of Bologna, Campus of Cesena}

  \textbf{Level}: A2\newline
  \textbf{Certification body}: Cambridge English Language Assessment
}
