\section{Research Projects and Initiatives}

\twocols{
  2024 - Today
}{
  \heading{DISCOV.ER}

  \textbf{Description}: DISCOV.ER is a research project focused on climate change monitoring and biodiversity conservation through the development of innovative technological solutions. Centered on the Po Delta Natural Park, the project involves the creation of a digital twin of key natural and tourist areas, integrating data from custom-designed sensor networks, external sources, and citizen science contributions (images, video, audio). Co-designed with stakeholders and local communities, the digital twin supports both expert analysis for environmental forecasting and public engagement through immersive experiences, promoting awareness of biodiversity and sustainable eco-tourism.

  \textbf{Role}: Member of the Research Team
}

\twocols{
  2023 - Today
}{
  \heading{CitizER Science}

  \textbf{Description}: CitizER fosters collaboration between institutions and citizens to co-create shared solutions to local challenges, supporting inclusive and participatory decision-making. Developed by the Emilia-Romagna Digital Agenda in partnership with ART-ER, the project is part of the 2020-2025 \textit{Data Valley Bene Comune strategy}. It aims to strengthen the regional ecosystem of Citizen Science by providing a conceptual framework, promoting best practices, and encouraging the growth of community-driven scientific initiatives through open data and active public engagement.

  \textbf{Role}: Tutor
}

\twocols{
  2023 - 2024
}{
  \heading{SMARTLAGOON}

  \textbf{Description}: SMARTLAGOON is a multidisciplinary research project focused on developing a digital twin of coastal lagoons to support data-driven decision-making for sustainable management. It combines environmental monitoring, socio-economic analysis, and advanced ICT tools, including AI, IoT, and system modeling, to simulate and predict the impact of human activity and climate change. The project aims to foster participatory governance and was funded by the European Union's Horizon 2020 program.

  \textbf{Role}: Research Fellow, Member of the Research Team
}

\twocols{
  2021
}{
  \heading{Navile}

  \textbf{Description}: AlmaMap Navile is an interactive mapping and wayfinding system developed for public displays at the University of Bologna's Navile campus. Commissioned by the Building and Sustainability Area (Area Edilizia e Sostenibilità - AUTC) and carried out by the Department of Computer Science and Engineering, the project aims to enhance campus navigation through intuitive digital maps and directional assistance.

  \textbf{Role}: Developer
}

\twocols{
  2020 - 2021
}{
  \heading{Maré}

  \textbf{Description}: Maré is a privacy-focused digital platform to support safe post-pandemic recovery through non-intrusive contact detection (e.g., via Wi-Fi). Designed with a citizen-centered approach, it enables localized testing and community engagement in collaboration with health professionals. The project was funded by FCT under the \textit{RESEARCH 4 COVID-19} initiative and coordinated by Nuno Jardim Nunes (Técnico Lisbon / Interactive Technologies Institute (ITI / LARSyS)).

  \textbf{Role}: Developer
}
