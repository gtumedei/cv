\documentclass[a4paper,10pt]{article}

% Packages
% -------------------- %

% Apply the document encoding
\usepackage[utf8]{inputenc}
% Set the document language/s
\usepackage[english]{babel}
% Set margins
\usepackage[hmargin=2cm,vmargin=3cm,headheight=16pt]{geometry}
% Apply the Merryweather font
\usepackage{merriweather}
\usepackage[T1]{fontenc}
% Customize links
\usepackage[english]{hyperref}
% Include graphics
\usepackage{graphicx}
% Create better tables
\usepackage{tabularx}
% Use floating positioning (like H inside figures)
\usepackage{float}
% Customize lists
\usepackage{enumitem}
% Customize the space between paragraphs
\usepackage{parskip}
% Provide additional color syntax
\usepackage{xcolor}
% Customize headers and footers
\usepackage{fancyhdr}
% Create table cells with line breaks inside
\usepackage{makecell}

% Package config
% -------------------- %

% Remove indent for the first paragraph line
\setlength{\parindent}{0pt}

% Remove table cell padding
\renewcommand{\tabcolsep}{1pt}

% Customize link colors
\hypersetup{
  colorlinks=true,
  linkcolor=black,
  filecolor=blue,
  urlcolor=blue,
  citecolor=blue
}

% Use dashes for itemize
\renewcommand\labelitemi{-}

% Set graphics folder path
\graphicspath{{./img/}}

% Heading color
\definecolor{sky}{HTML}{2079C7}

% Setup page styles

% Page number only
\fancypagestyle{style1}{
  \fancyhf{}
  \renewcommand{\headrulewidth}{0pt}
  \fancyfoot[R]{\thepage}
}

% Name and page number
\fancypagestyle{style2}{
  \fancyhf{}
  \renewcommand{\headrulewidth}{0pt}
  \fancyhead[R]{\name}
  \fancyfoot[R]{\thepage}
}

\newcommand{\name}{Gianni Tumedei}

\renewcommand{\title}[1]{{\huge \textbf{#1} \vspace{20pt}}}

\renewcommand{\section}[1]{{\sffamily
  \vspace{10pt}
  \textcolor{sky}{\textbf{\expandafter\MakeUppercase\expandafter{#1}}}
  \vspace{20pt}
}}

\newcommand{\heading}[1]{{\large \textbf{#1} \vspace{4pt}}}

\newcommand{\subheading}[1]{\textit{#1}\newline}

\newcommand{\twocols}[2]{
  \begin{tabularx}{\textwidth}{ p{0.3\textwidth} X p{0.65\textwidth} }
  % \begin{tabularx}{\textwidth}{|p{0.3\textwidth}|X|p{0.65\textwidth}|}
    #1 && #2 \\
  \end{tabularx}
  \vspace{20pt}
}

\newcommand{\link}[2]{\underline{\href{#1}{#2}}}

\newcommand{\email}[1]{\link{mailto:#1}{#1}}
\newcommand{\tel}[1]{\link{tel:#1}{#1}}

\newcommand{\iconrow}[2]{
  \raisebox{\dimexpr-\height+\baselineskip-2pt}{#1} & #2 \\
}

\input{env}

\begin{document}

\pagestyle{style1}

\title{\name}

\section{Publications}

\twocols{
  2025
}{
  \heading{Enhancing Building Environments: A Digital Twin Approach for Informed Decision-Making and Better Campus Experiences}
  \newline
  \subheading{Gianni Tumedei, Chiara Ceccarini, Luca Giulianini Giovanni Delnevo, Catia Prandi}

  Information and Software Technology.

  \link{https://doi.org/10.1016/j.infsof.2026.108044}
}

\twocols{
  2025
}{
  \heading{An Integrated Environmental and Perceptual Dataset for Predicting Comfort in Smart Campuses During the Fall Semester}
  \newline
  \subheading{Gianni Tumedei, Chiara Ceccarini, Giovanni Delnevo, Catia Prandi}

  Data.

  \link{https://doi.org/10.3390/data11020031}
}

\twocols{
  2025
}{
  \heading{\textit{DISCOV.ER} the Biodiversity of a Unique Wetland Site Combining Citizen Science, Gamification, and Generative AI}\newline
  \subheading{Gianni Tumedei, Riccardo Gardenghi, Silvia Mirri, Catia Prandi}

  CCNC 2026: 2nd International Workshop on Digital Sustainability for Comsumer Applications: Innovations in Communications and Networking (DISCA'26)

  Workshops and Affiliated Events

  \link{https://doi.org/10.1109/CCNC65079.2026.11366474}
}

\twocols{
  2025
}{
  \heading{DISCOV.ER: Technology and Citizen Science for the Sustainability and Attractiveness of Areas of Naturalistic and Touristic Interest}\newline
  \subheading{Gianni Tumedei, Catia Prandi, Elio Amadori, Serena Tedeschi, Francesco Maria Riccio}

  In Proceedings of the ACM International Conference on Information Technology for Social Good (GoodIT'25).

  Work-in-Progress Track

  \link{http://doi.org/10.1145/3748699.3749823}
}

\twocols{
  2025
}{
  \heading{Augmenting the Intangible: an Intervention Through an AR Mobile App and an Installation to Foster a Local Cultural Phenomenon}\newline
  \subheading{Catia Prandi, Chiara Ceccarini, Gianni Tumedei, Paola Salomoni}

  Frontiers in Computer Science.

  \link{https://doi.org/10.3389/fcomp.2025.1629965}
}

\twocols{
  2025
}{
  \heading{From Drawings to Awareness: Exploring Narrative Visualization and AI to Teach Children About the Fragile Ecosystem of the Mar Menor Lagoon}\newline
  \subheading{Gianni Tumedei, Chiara Ceccarini, Inmaculada Concepción Jimenez Navarro, Catia Prandi}

  In proceedings of the ACM Designing Interactive Systems Conference (DIS) 2025.

  \link{https://doi.org/10.1145/3715336.3735722}
}

\twocols{
  2025
}{
  \heading{\textit{How are you, Mar Menor?} Fostering Awareness About an Ecological Crisis through Children's Art and Conversational Generative AI}\newline
  \subheading{Gianni Tumedei, Chiara Ceccarini, Catia Prandi}

  In proceedings of the Nineteenth International Conference on Tangible, Embedded, and Embodied Interaction (TEI '25).

  Work-in-Progress Track

  \link{https://doi.org/10.1145/3689050.3705984}
}

\twocols{
  2024
}{
  \heading{Transforming Smart Campuses into User-Centric Environments by integrating BIM and Environmental Data}\newline
  \subheading{Gianni Tumedei, Chiara Ceccarini, Catia Prandi}

  In proceedings of TrustSense 2024: 1st International Workshop on Pervasive Computing Challenges in Trustable Crowdsensing Systems - 2024 IEEE International Conference on Pervasive Computing and Communications (PERCOM 2024).

  Workshops and Affiliated Events

  \link{https://doi.org/10.1109/PerComWorkshops59983.2024.10502582}
}

\twocols{
  2024
}{
  \heading{Toward a Digital Twin: combining sensing, machine learning, and data visualization for the effective management of a coastal lagoon environment}\newline
  \subheading{Giovanni Delnevo, Gianni Tumedei, Vittorio Ghini, Catia Prandi}

  In proceedings of CCNC 2024: 3rd International workshop on IoT interoperability and the web of things (IIWOT'24).

  Track 2 - Networking Solutions for Social Applications, Multimedia, and Games

  \link{https://doi.org/10.1109/CCNC51664.2024.10454647}
}

\twocols{
  2024
}{
  \heading{On developing a procedural level generator based on the Model Synthesis algorithm in the context of serious games}\newline
  \subheading{Davide Paolillo, Bryan Corradino, Gianni Tumedei, Mariagrazia Benassi, Catia Prandi}

  In proceedings of CCNC 2024: 3rd International workshop on IoT interoperability and the web of things (IIWOT'24).

  Track 2 - Networking Solutions for Social Applications, Multimedia, and Games

  \link{https://doi.org/10.1109/CCNC51664.2024.10454803}
}

\twocols{
  2024
}{
  \heading{Train your attention and executive functions with Eye-Riders! A videogame for improving cognitive abilities in neurodiverse children}\newline
  \subheading{Mariagrazia Benassi, Davide Paolillo, Matilde Spinoso, Sara Giovagnoli, Noemi Mazzoni, Luca Formica, Gianni Tumedei, Catia Prandi}

  In proceedings of CCNC 2024: 3rd International workshop on IoT interoperability and the web of things (IIWOT'24).

  Track 2 - Networking Solutions for Social Applications, Multimedia, and Games

  \link{https://doi.org/10.1109/CCNC51664.2024.10454866}
}

\twocols{
  2023
}{
  \heading{Evaluating the use of machine learning algorithms in environmental sensing for energy saving}\newline
  \subheading{Giovanni Delnevo, Gianni Tumedei, Vittorio Ghini, Catia Prandi}

  In Proceedings of NET4us '23 (The 2nd Workshop on Networked Sensing Systems for a Sustainable Society, pp 201-206).

  \link{https://doi.org/10.1145/3615991.3616402}
}

\twocols{
  2023
}{
  \heading{What Does Air Quality Sound Like? On Exploring the impact of Data Sonification Versus Data Visualization}\newline
  \subheading{Chiara Ceccarini, Gianni Tumedei, Gianluca Migliarini, Catia Prandi}

  In Proceedings of GoodIT'23 (ACM International Conference on Information Technology for Social Good, pp 510-516).

  Track: GoodIT'23 - Work-in-Progress Track Papers

  \link{https://doi.org/10.1145/3582515.3609575}
}

\twocols{
  2022
}{
  \heading{Towards a Smart Campus Digital Twin: Promoting Awareness and Sustainability Through Wayfinding and Real-Time Environmental Data}\newline
  \subheading{Gianni Tumedei, Catia Prandi (supervisor), Augusto Esteves (co-supervisor), Andrea Manzo (co-supervisor)}

  \link{https://amslaurea.unibo.it/27643}
}

\twocols{
  2022
}{
  \heading{Crowdsensing-enabled Service Design for Floating Students during the COVID-19 Pandemic}\newline
  \subheading{Shuhao Ma, Valentina Nisi, Augusto Esteves, Catia Prandi, Hugo Nicolau, Gianni Tumedei, João Nogueira, Francesco Boschi, and Nuno Jardim Nunes}

  In Congress of the International Association of Societies of Design Research (IASDR 2021, pp 943-959). Springer, Singapore.

  \link{https://doi.org/10.1007/978-981-19-4472-7\_61}
}

\twocols{
  2021
}{
  \heading{Promoting a Safe Return to University Campuses during the COVID-19 Pandemic: Crowdsensing Room Occupancy}\newline
  \subheading{Gianni Tumedei, Francesco Boschi, Catia Prandi, Luìs Gomes, Rui Calheno, Rui Abreu, Shuhao Ma, Valentina Nisi, Nuno Nunes, Augusto Esteves}

  In Proceedings of the Conference on Information Technology for Social Good (GoodIT 2021, pp 145-150).

  \link{https://doi.org/10.1145/3462203.3475911}
}

\twocols{
  2019
}{
  \heading{Il protocollo OPC UA}\newline
  \subheading{Gianni Tumedei, Franco Callegati (supervisor)}

  \link{https://amslaurea.unibo.it/19780}
}


\end{document}
